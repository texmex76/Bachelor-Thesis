\section{Outlook / Conclusion}

In this thesis, we asked ourselves the question how to arrange atoms in a lattice to simulate a solid. The reason we might want to do that is that on top of a naturally occurring solid being hard to look into, we can't change it, e.g. modify the lattice structure. Since atoms arranged in a lattice are able to represent a quantum system under certain conditions, we can use that quantum system to simulate another. Here we're stepping in the rapidly expanding field of quantum simulations. For those wishing to explore some of the recent advancements, \cite{quantum_sanchez} is part of a dossier with mini-reviews. In our case, we make use of light to trap atoms, more specifically lasers. If two laser beams are counter-propagating, the intensity will form a cosine pattern. Between those two lasers, there's our atoms. The special trick is to set the frequency of the lasers way below the excitation frequency of the atoms. That will induce a dipole in the atoms which will interact with the light field. Now we have a potential for the atoms and they will go to the lowest points to minimize their energy. To enhance atom-light field interaction, we put the atoms in a cavity. We looked at two ways to pump the cavity: longitudinally and transversally. The transversal pump has a couple special properties, one of which is the sudden phase transition: Initially, nothing happens when we pump and the atoms stay uniformly distributed. Then, at a critical pump strength, a light field will suddenly build up and the atoms be arranged in a lattice. In the programming language Julia with the framework QuantumOptics.jl we conducted a couple simulations which fulfilled our expectations quite neatly.