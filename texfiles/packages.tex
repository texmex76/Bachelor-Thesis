\usepackage{relsize}
\usepackage[margin=3cm]{geometry} %Definiere Rand
\usepackage{graphicx} % Zum Einbinden von Bildern
\usepackage[english]{babel} % Direkte Eingabe von Umlauten
%\usepackage[utf8]{inputenc} % Direkte Eingabe von Umlauten
\UseRawInputEncoding
\usepackage[T1]{fontenc}  % Direkte Eingabe von Umlauten
\usepackage{pgfplots} % Zum Einfuegen von Plots
\pgfplotsset{compat=1.14} % Damit wird beim Plotten keinen Error bekommen
\usepackage[section]{placeins} %Damit Bilder in der Section bleiben
\usepackage{amsmath} % Standard fuer mathematische Ausdruecke
\usepackage{amssymb} % Weitere Symbole
\usepackage{mathtools} % Fuer weitere mathematische Ausdruecke
\usepackage{siunitx} % Um SI-Einheiten anzugeben
\usepackage[font=small,labelfont=bf]{caption} % Kleinerer Text bei Captions
\usepackage{tabu} % Anderes Tabellenenvironment, wird am Ende fuer die Namen verwendet
\usepackage{subcaption} % Side-by-side figures with minipage
\usepackage{url} % Damit man urls zitieren kann
\usepackage[autostyle=true,german=quotes]{csquotes} % Damit Zitieren leichter ist; Bsp: \enquote{nur}
\usepackage[nottoc,numbib]{tocbibind} % Damit die Referenzen im Inhaltsverzeichnis erscheinen
% Die folgenden zwei Zeilen benennen "Inhaltsverzeichnis" zu "Referenzen um
\usepackage{standalone} % Zum Outsourcen von Plots
\addto\captionsngerman{\renewcommand{\bibname}{Referenzen} \renewcommand{\refname}{Referenzen}} % Benennt "Literatur" in "Referenzen" um
\sisetup{range-phrase=-} % Wie SI-Intervalle angezeigt werden
\usepackage{setspace} % Damit die naechsten funktionieren
\renewcommand{\topfraction}{0.85} % Let top 85% of a page contain a figure
\renewcommand{\textfraction}{0.1} % Default amount of minimum text on page (Set to 10%)
\renewcommand{\floatpagefraction}{0.75} % Only place figures by themselves if they take up more than 75% of the page
\DeclareSIPostPower\tothefourth{4} % SI-Einheiten zur 4. Potenz
%
\makeatletter
\newcommand*{\rom}[1]{\expandafter\@slowromancap\romannumeral #1@}
\makeatother %roman numerals
\usepackage{threeparttable, tablefootnote}
\DeclareSIUnit\torr{torr}
\usepackage{circuitikz} %For circuits
\renewcommand{\d}{\mathrm{d}}
\newcommand{\intl}{\int\limits}
\sisetup{math-micro=\text{µ},text-micro=µ}
\usepackage{hyperref} % For hyperlinks to be clickable

\usepackage{listings} % For embedding code into the document
\usepackage{jlcode} % Add julia syntax to listings package
\lstset{language=Julia}